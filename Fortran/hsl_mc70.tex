\documentclass{stfc}
\usepackage{algorithm}
\usepackage{algorithmic}

% set the release and package names

\newcommand{\libraryname}{HSL}
\newcommand{\packagename}{MC70}
\newcommand{\fullpackagename}{\libraryname\_\packagename}
\newcommand{\versionum}{1.0.0}
\newcommand{\version}{\versionum}
\begin{document}

\hslheader

\hslsummary

Let $A$ be an $n \times n$ matrix with a symmetric sparsity pattern.
{\tt HSL\_MC70} computes a {\bf nested dissection ordering} of $A$
that is suitable for use with a sparse direct solver. The 
algorithm allows for some dense or nearly dense rows and columns in $A$.

The algorithm partitions the rows/columns of $A$ into 3 sets such that 
reordering the rows/columns to respect to their partitions yields a symmetric 
matrix of the form
\begin{equation}\label{eqn:partition}
\left(\begin{array}{c|c|c} A_1 & 0 & S_1^T \\\hline 0 & A_2 & S_2^T \\\hline S_1 & S_2 & S \end{array}   \right).
\end{equation}
The rows of $S$ are given an arbitrary order.
If the dimension of $A_1$ ($A_2$) is smaller than some predefined value, the rows 
of $A_1$ ($A_2$) will be ordered using an approximate minimum degree algorithm; 
otherwise, the rows/columns of $A_1$ ($A_2$) will be partitioned to form 
another matrix with the above structure and the algorithm will continue to be 
applied in a recursive manner.




\hslattributes
\hslversions{\versionum}.
\hslIRDCZ Real (single, double).
\hsluses {\tt HSL\_FA14}, {\tt HSL\_MC78} and {\tt HSL\_MC79}.
\hsllanguage Fortran~2003 subset (F95+TR155581)
\hsldate January 2014.
\hslorigin I.S. Duff, J.A. Scott and H.S. Thorne, Rutherford
Appleton Laboratory.

\hslhowto

\subsection{Calling sequences}


Access to the package requires a {\tt USE} statement of the form


\hspace{8mm}{\tt USE  \fullpackagename\_integer}



%\noindent
%The following procedures are available to the user:
%\begin{description}
%\item (a)  To compute a nested dissection ordering,
%{\tt MC70\_nested} should be called.
%\vspace{-0.1cm}
%\item (b) {\tt MC70\_print\_message} can be used after a return from
%{\tt MC70\_nested} to print the error 
%message associated with a nonzero error flag.
%\end{description}

%%%%%%%%%%%%%%%%%%%%%%%%%%%%%%%%%%%%%%%%%%%%%%%%%
\hsltypes
\label{derived types}
For each problem, the user must employ the derived types defined by the
module to declare scalars of the types
{\tt MC70\_control} and {\tt MC70\_info}. 
The following pseudocode illustrates this.
\begin{verbatim}
      use HSL_MC70_integer    
      ...
      type (MC70_control) :: control       
      type (MC70_info) :: info
      ...
\end{verbatim}
The components of {\tt MC70\_control} and {\tt MC70\_info} are explained
in Sections~\ref{typecontrol} and \ref{typeinform}.
 

%%%%%%%%%%%%%%%%%%%%%%%%%%%%%%%%%%%%%%%%%%

\section{THE ARGUMENT LISTS\label{ArgLists}}


\subsection{Input of the matrix $A$}
The user must supply the pattern of either the {\bf lower triangular part} of the matrix $A$ 
or the {\bf lower and upper triangular} part of $A$ 
in a compressed sparse column format. There is 
no requirement that zero entries on the diagonal are explicitly included. 
{\bf No checks} are made on the user's data. It is important to note
that any out-of-range entries or duplicates may cause {\tt HSL\_\packagename}
to fail in an unpredictable way. Before using {\tt HSL\_\packagename},
the HSL package {\tt HSL\_MC69} may  be used to check for errors and to handle
duplicates ({\tt HSL\_MC69} sums them) and out-of-range entries 
({\tt HSL\_MC69} removes them).

If the user's data is held using
another standard sparse matrix format (such as coordinate format
or sparse compressed row format), we recommend using
a conversion routine from {\tt HSL\_MC69} to put the data into the required format. 
The input of $A$ is illustrated in Section~\ref{hslexample}.


\subsection{To compute a nested dissection ordering}
If the user has the {\bf lower triangular part of} $A$ held in 
compressed sparse columns format, a call 
of the following form should be made:
\subroutine{mc70\_order(n,ptr,row,perm,control,info)}
\vspace{-1.0em}
If the user has the {\bf lower and upper triangular parts of} $A$ held in 
compressed sparse columns format, a call of the following form should be made:
\subroutine{mc70\_order\_full(n,ptr,row,perm,control,info)}
\vspace{-1.0em}
\begin{description}
\itt{n} is an {\tt INTEGER} scalar with {\tt INTENT(IN)}.  On entry it
must hold the order $n$ of $A$.
{\bf  Restriction:} {\tt n $\ge$ \tt 1}.

\itt{ptr} is an {\tt INTEGER} array of rank one with {\tt INTENT(IN)}
and size {\tt n+1}. It must be set by the user so that
{\tt ptr(j)} is the position in {\tt row} of the first entry in column
{\tt j} ({\tt j=1,2,...,n}) and {\tt ptr(n+1)} must be set to one more than
the total number of entries.

\itt{row} is an {\tt INTEGER} array of rank one with {\tt INTENT(IN)}
and size  {\tt ptr(n+1)-1}. On a call to {\tt mc70\_order}, it must be 
set by the user so that {\tt row(1:ptr(n+1)-1)} holds the row indices of the entries in 
the {\bf lower triangular part} of $A$; on a call to 
{\tt mc70\_order\_full}, it must be 
set by the user so that {\tt row(1:ptr(n+1)-1)} holds the row indices of the entries in 
the {\bf lower and upper triangular parts} of $A.$ The entries in a single 
column must be contiguous. The entries of column {\tt j} must precede those of 
column {\tt j+1} {\tt (j=1,2,\ldots,n-1)}, and there must be no wasted space 
between the columns. Row indices within a column may be in any order. Diagonal 
entries are ignored.

\itt{perm} is an {\tt INTEGER} array with {\tt INTENT(OUT)} and size
{\tt n}.  On exit, {\tt perm} holds the nested dissection ordering. 
The position of variable {\tt i} in the nested dissection ordering is  {\tt perm(i)},
{\tt i=1,2,\ldots,n}.


\itt{control}  is a scalar of type {\tt MC70\_control} with {\tt
INTENT(IN)}.  Its components control the action, as explained in
Section~\ref{typecontrol}.

\itt{info}  is a scalar of type {\tt MC70\_info} with {\tt
INTENT(OUT)}.  Its components hold information, as explained in
Section~\ref{typeinform}.


\end{description}

%%%%%%%%%%%%%%%%%%%%%%%%%%%%%%%%%%%%%%%%%%%%%%%%%%%%%%%%%%%%%%

%\subsection{printing error messages\label{MatrixprintMsg}}

%Subroutine {\tt MC70\_print\_message} can be used after return from
%{\tt MC70\_nested} 
%to print an error message associated with a nonzero
%error flag {\tt info\%flag}.

%\subroutine{MC70\_print\_message(flag,[,unit,message])}
%\vspace{-1.0em}

%\begin{description}

%\itt{flag} is an {\tt INTEGER} scalar of {\tt INTENT(IN)}. On entry,
%       it must be set by the user
%	to hold the error flag
%	generated when calling {\tt MC70\_nested}.

%\itt{unit} is an {\tt OPTIONAL} {\tt INTEGER} scalar of
%	{\tt INTENT(IN)}.
%	If present, on entry, it must be set by the user to hold the
%	unit number for printing the message. If this number
%	is negative, printing is suppressed. If {\tt unit} is not
%	present, the message will be printed on unit {\tt 6}.

%\itt{message} is an {\tt OPTIONAL} {\tt CHARACTER (LEN=*)} scalar
%	with {\tt INTENT(IN)}. If present, on entry, it must be set by
%	the user to hold the message to be printed
%	ahead of the error message.
%\end{description}

%%%%%%%%%%%%%%%%%%%%%%%%%%%%%%%%%%%%%%%%%%%%%%%%%%%%%

\subsection{The control derived data type}
\label{typecontrol}
The derived data type {\tt MC70\_control} is used to control the
action.  The user must declare
a structure of type {\tt MC70\_control}.
 The components, which are automatically 
given default values in the definition of the type, are: \\

\noindent {\bf Printing controls}

\begin{description}
\itt{print\_level} is a scalar of type  {\tt INTEGER}
that is used to controls the level of  printing. The different levels are:
\begin{description}
\item{\tt $<$ 0 } No printing.
\item{\tt $=$ 0 } Error messages only.
\item{\tt $=$ 1 } As 0, plus basic diagnostic printing.
\item{\tt $>$ 1 } As 1, plus some additional diagnostic printing.
\end{description}
The default is {\tt print\_level$=$\tt 0}.

\itt{unit\_diagnostics} is a scalar  of type  
{\tt INTEGER} that holds the
unit number for diagnostic printing. Printing is suppressed if 
{\tt unit\_diagnostics$<0$}.
The default is {\tt unit\_diagnostics$=$6}.

\itt{unit\_error} is a scalar of type  {\tt INTEGER} that holds the
unit number for error messages.
Printing of error messages 
is suppressed if {\tt unit\_error$<$0}. 
The default is {\tt unit\_error$=$6}.

\end{description}

\noindent {\bf Other controls}


\begin{description}

%\itt{ml} is an {\tt INTEGER} scalar used to determine whether a multilevel 
%partitioning scheme is used.  The different options are
%\begin{description}
%\item{\tt $<$ 1 } Multilevel partitioning scheme is {\bf not} used,
%\item{\tt $=$ 1 } Multilevel partitioning scheme is  used,
%\item{\tt $>$ 1 } Automatic choice of whether the multilevel partitioning scheme is used.
%\end{description}
%The multilevel partitioning scheme coarsens the matrix (reduces its order 
%whilst trying to maintain some of the structural properties) recursively. Once 
%the coarsened matrix is of order at most {\tt ml\_switch} or the maximum 
%number of multilevel recursions {\tt ml\_max\_levels} has been reached, the coarsened 
%matrix is partitioned (the partitioning method used is determined by 
%{\tt ml\_partition\_method}. This partition is projected (and refined) 
%up through the hierarchy of multilevel matrices. 
%The recursive multilevel partitioning method will 
%also stop coarsening if the next coarse matrix would be too small or too large 
%relative to the current coarse matrix, see {\tt ml\_max\_reduction} and 
%{\tt ml\_min\_reduction}. If the multilevel partitioning scheme is not 
%used, the matrix is partitioned using the method defined by 
%{\tt ml\_partition\_method}. See Section~\ref{sec:ml} for further details 
%about multilevel partitioning. The default is {\tt ml=2}. 

%\itt{ml\_bandwidth} is a {\tt DOUBLE PRECISION} scalar and is only used if 
%{\tt ml$>$1}. Let $A$ be the matrix for which we wish to find a nested 
%dissection ordering, $\widetilde{A}$ be the matrix that results from 
%(optionally) removing any dense rows and compressing the matrix, and 
%{\tt ml$>$1}. Suppose that the Reverse Cuthill-McKee algorithm applied to 
%$\widetilde{A}$  yields a matrix with bandwidth $w.$ If 
%$w\ge\tilde{n}\times{\tt ml\_bandwidth},$ the multilevel partitioning 
%method is used within the nested dissection algorithm; otherwise, the 
%multilevel partitioning method is not used. The default is 
%{\tt ml\_bandwidth=0.01}.

%\itt{ml\_max\_levels} is an {\tt INTEGER} scalar that holds the maximum number
%of multilevel recursions, see {\tt ml}.  If 
%{\tt ml\_max\_levels} $\le$ {\tt 0}, the multilevel partitioning scheme is not used. 
%The default is {\tt ml\_max\_levels=20}.

%\itt{ml\_max\_reduction} is a {\tt DOUBLE PRECISION} scalar. During the multilevel 
%partitioning scheme, suppose that the current coarse matrix has order $n_f$ and 
%the subsequent coarsened matrix would have order $n_c.$ If 
%$n_c>n_f\times{\tt ml\_max\_reduction},$ the multilevel recursion will 
%terminate and the current coarse matrix will be 
%partitioned using the method defined by {\tt partition\_method}. This 
%partition is projected (and refined) up through the hierarchy of multilevel 
%matrices. Values less than {0.5} are treated as {\tt 0.5}. Values greater than 
%{\tt 1.0} are treated as {\tt 1.0}. The default is {\tt ml\_max\_reduction=0.9}.

%\itt{ml\_min\_reduction} is a {\tt DOUBLE PRECISION} scalar. During the multilevel 
%partitioning scheme, suppose that the current coarse matrix has order $n_1$ and 
%the subsequent coarsened matrix would have order $n_2.$ If 
%$n_c<n_f\times{\tt ml\_min\_reduction},$ the multilevel recursion will 
%terminate and the current coarse matrix will be 
%partitioned using the method defined by {\tt partition\_method}. This 
%partition is projected (and refined) up through the hierarchy of multilevel 
%matrices. Values less than 
%{\tt 0.01} are treated as {\tt 0.01}. The default is {\tt ml\_min\_reduction=0.01}.


%\itt{ml\_switch} is an {\tt INTEGER} scalar that is used in the criteria for 
%determining when to terminate the recursive multilevel partitioning scheme and 
%partition the current coarse matrix, see {\tt ml}.   Values less
%than {\tt 2} are treated as {\tt 2}.
%The default is {\tt ml\_switch=50}.


\itt{max\_levels} is an {\tt INTEGER} scalar that holds the maximum number
of nested dissection levels. If the maximum number of nested dissection levels 
is reached, the corresponding matrix in the nested dissection hierarchy 
will be ordered using an approximate minimum degree algorithm. If 
{\tt max\_levels = 0}, the ordering of the input matrix is computed 
using an approximate minimum degree algorithm.  Values less
than {\tt 0} are treated as {\tt 0}.  The default is {\tt max\_levels=20}.



\itt{nd\_switch} is an {\tt INTEGER} scalar that is used in the criteria for 
determining what happens a matrix within the nested dissection hierarchy. If 
the matrix order is greater than {\tt nd\_switch} and the maximum number of 
nested dissection levels has not been reached, the matrix will be 
partitioned; otherwise, it will be ordered using an approximated minimum 
degree algorithm. Values less than {\tt 2} are treated as {\tt 2}.  The 
default is {\tt nd\_switch=50}.


%\itt{partition\_method} is an {\tt INTEGER} scalar that determines  method 
%used to partition a matrix (the coarse matrix if using the multilevel
%partitioning scheme). The different options are:
%\begin{description}
%\item{\tt $\le$ 1 } Partition using the half-level set method.
%\item{\tt $\ge$ 2 } Partition using the level-set method.
%\end{description}
%See Section~\ref{sec:ml} for further details. The default is {\tt
%partition\_method=1}. 

\itt{ratio} is a {\tt DOUBLE PRECISION} scalar. The partitioning and refinement
methods aim to find partitions (\ref{eqn:partition}) such that
$\max(n_1,n_2)<\min(n_1,n_2)\times{\tt ratio},$ where $A_1$ has order $n_1$ and
$A_2$ has order $n_2.$ If several candidate partitions satisfy this requirement,
then the partition with the smallest value of $\frac{n_s}{n_1 n_2}$ is chosen,
where $S$ has order $n_s;$ if none of the candidate partitions satisfy this
requirement, the partition that has the smallest value of
$\frac{n_s}{n_1 n_2}$ is chosen. Decreasing {\tt ratio} will, in
general, result in a nested dissection ordering that is more amenable to
parallel direct solvers; increasing {\tt ratio} will, in general, reduce the
number of non-zeros in the Cholesky factorization of the reordered matrix. The
default is {\tt ratio=1.5}.

%\itt{refinement} is an {\tt INTEGER} scalar that determines the method used to
%refine a partition. Given a matrix partition of the form (\ref{eqn:partition}),
%the partition is modified with the aim of reducing the order of $S$ and
%balancing the orders of $A_1$ and $A_2.$ The different options are
%\begin{description}
%\item{\tt $<$ 1 } Trimming algorithm followed by Fiduccia-Mattheyses algorithm,
%\item{\tt $=$ 1 } Dulmage-Mendelsohn algorithm followed by Fiduccia-Mattheyses algorithm,
%\item{\tt $>$ 1 } Automatic choice of refinement method based on partition properties.
%\end{description}
%See Section~\ref{sec:ref} for further details about these methods. The default 
%is {\tt refinement=2}. 

%\itt{refinement\_band} is an {\tt INTEGER} scalar used within the 
%Fiduccia-Mattheyses refinement. The variables in the refined separator must be 
%at most distance {\tt refinement\_band} from a variable that was in the input 
%separator. See Section~\ref{sec:ref} for further details. In general, small 
%values of {\tt refinement\_band} result in a faster method but the quality of 
%the ordering may be compromised. The default is {\tt refinement\_band=4}.

\itt{remove\_dense} is a {\tt LOGICAL} scalar. If {\tt remove\_dense = .true.}, 
then the input matrix is searched for dense (or nearly dense) rows and columns, 
and the nested dissection algorithm is applied to the matrix that results when 
these rows and columns are removed. Dense rows/columns are placed at the end of 
the ordering. If {\tt remove\_dense = .false.}, then the input matrix is not 
searched for dense rows. The default is {\tt remove\_dense=.true.}.


\end{description}


%%%%%%%%%%% inform type %%%%%%%%%%%

\subsection{The derived data type for holding information}
\label{typeinform}
The derived data type {\tt MC70\_info} 
is used to hold parameters that give information about 
the algorithm. The components of {\tt MC70\_info} 
(in alphabetical order) are:

\begin{description}

\itt{flag} is a scalar of type  {\tt INTEGER}
that gives the exit status of the algorithm (details in Section \ref{hslerrors}).

\itt{ndense} is a scalar of type {\tt INTEGER} that holds 
the number of rows/columns in the input matrix that were determined to be dense. 
If {\tt control\%remove\_dense = .false.}, the input matrix is not checked for 
dense rows.

\itt{stat} is a scalar of type  {\tt INTEGER}
that holds the Fortran {\tt stat} parameter. 
\end{description}



\subsection{Error diagnostics\label{hslerrors}}

On successful completion, {\tt MC70\_nested} 
will exit with  {\tt info\%flag} set to {\tt 0}. Other values
for {\tt info\%flag} are associated with a fatal error. 
Possible values are:
\vspace{-0.1cm}
\begin{description}
\item{} {\tt -1} memory allocation failed.
\item{} {\tt -2} memory deallocation failed.
\item{} {\tt -3} {\tt n} $\le$ 0.

\end{description}



\hslgeneral

\hslworkspace Provided automatically by the module.

\hslmodules  {\tt HSL\_FA14}, {\tt HSL\_MC78} and {\tt HSL\_MC79}.
\hslio Error and diagnostic messages.
   Error messages on unit {\tt control\%unit\_err}
   and diagnostic messages on unit {\tt control\%unit\_diagnostics}. These have default value {\tt 6};
   printing of these messages is suppressed if the relevant unit number
   is negative or if {\tt control\%print\_level} is negative.

\hslrestrictions
{\tt n $\ge$ 1}.
\hslportability Fortran~2003 subset (F95+TR155581)

\hslmethod



Given a symmetric matrix $A$ of order $n,$ {\tt \fullpackagename} preprocesses the matrix to (optionally) remove dense or almost dense rows/columns, returning $\overline{A}.$ The matrix $\overline{A}$ is then compressed using {\tt HSL\_MC78} to give a symmetric matrix $\widetilde{A}$ with order $\tilde{n}.$ This matrix is passed to the recursive nested dissection algorithm, Algorithm~\ref{alg:nd}, and the resulting permutation matrix is converted into the corresponding elimination ordering. Having formed a nested dissection ordering for $\widetilde{A},$ the ordering is mapped to give an ordering for $\overline{A}.$ The dense rows are appended to the end of the ordering. 

\begin{algorithm}\caption{Nested dissection algorithm}\label{alg:nd}
\begin{algorithmic}
\STATE {\bf recursive subroutine nested\_dissection}($A,P$)
\STATE Input: symmetric matrix $A$ of order $n$
\STATE Output: permutation matrix $P$

\IF{$n<${\tt nd\_switch}}
\STATE Form the AMD elimination ordering, $p,$ for $A$ and return its equivalent permutation matrix 
\ELSE
\STATE Partition the matrix: compute $P,$ a permutation matrix, such that $P^T A P$ has the form (\ref{eqn:partition})
\STATE Call {\bf nested\_dissection}($A_1,P_1$)
\STATE Call {\bf nested\_dissection}($A_2,P_2$)
\STATE Perform the update $P = PQ,$ where $$Q = \left( \begin{array}{ccc} P_1 & & \\ & P_2 & \\ & & I  \end{array} \right)$$
\ENDIF

\end{algorithmic}
\end{algorithm}

\subsection{Partitioning a matrix}
%Assume that $\widetilde{A}$ is irreducible. Given variables
%$u,v\in\{1,\ldots,\tilde{n} \},$ we define the path between $u$ and $v$ as being
%a sequence of variables, $l_0,\ldots,l_k\in \{1,\ldots,\tilde{n}\},$ subject to
%$l_0=u,$ $l_k=v$ and $\widetilde{A}(i,i+1)\neq 0$ for all $i=0,\ldots,k-1:$ this
%path has length $k.$ The distance between $u$ and $v$ is the length of the
%shortest path between $u$ and $v.$ Given a variable
%$u\in\{1,\ldots,\tilde{n}\},$ the level-set $L_i(u)$ is defined as the set of
%variables that are distance $i$ from $u.$

%At the heart of the partitioning methods, the 

%{\tt \fullpackagename} provides the user with a number of options for choosing the partitioning method. T



\subsection{Refinement}\label{sec:ref}

\hslexample


\end{document}


